% CVPR 2026 Paper Template
\documentclass[10pt,twocolumn,letterpaper]{article}

%%%%%%%%% PAPER TYPE
\usepackage[pagenumbers]{cvpr}      

% Additional packages
\usepackage{amsmath}
\usepackage{amssymb}
\usepackage{booktabs}
\usepackage{graphicx}
\usepackage{multirow}
\usepackage{subcaption}
\usepackage{xcolor}

% Custom commands for placeholders
\newcommand{\placeholder}[1]{\textcolor{red}{\textbf{[#1]}}}

\definecolor{cvprblue}{rgb}{0.21,0.49,0.74}
\usepackage[pagebackref,breaklinks,colorlinks,allcolors=cvprblue]{hyperref}

%%%%%%%%% PAPER ID
\def\paperID{*****} 
\def\confName{CVPR}
\def\confYear{2026}

%%%%%%%%% TITLE
\title{Implementing Transformers: From Attention Mechanisms to Grouped Query Attention}

%%%%%%%%% AUTHORS
\author{Abtin Pourhadi\\
Heinrich-Heine-Universität Düsseldorf\\
{\tt\small abtin.pourhadi@hhu.de}
}

\begin{document}
\maketitle

% SOURCE B STRUCTURE: No abstract block. Information integrated into Introduction.

%%%%%%%%% BODY TEXT
\section{Introduction}
\label{sec:intro}

The Transformer architecture, introduced by Vaswani et al.~\cite{vaswani2017attention}, revolutionised natural language processing by replacing recurrent mechanisms with self-attention mechanisms that enable flexible modeling of long-range dependencies.

This report presents a comprehensive implementation of the Transformer architecture developed throughout the course. We detail the construction of key components including BPE tokenization, scaled dot-product attention, and the complete encoder-decoder architecture. We establish a strong baseline by training a vanilla Transformer on German-to-English translation.

Building on this baseline, we extend the architecture by implementing Grouped Query Attention (GQA)~\cite{ainslie2023gqa} to optimize memory bandwidth and inference efficiency. Our experiments demonstrate that GQA ($G=2$) reduces KV cache memory by 75\% while maintaining translation quality comparable to the baseline.

\noindent\textbf{Contributions.}
\begin{itemize}
    \item Implementation of a baseline encoder-decoder Transformer with BPE tokenization, masked self-attention, and cross-attention.
    \item Empirical evaluation of the baseline on the WMT17 German-English dataset, including analysis of training dynamics.
    \item Implementation and analysis of Grouped Query Attention (GQA) as an architectural extension to improve inference efficiency.
\end{itemize}
All code used for the implementation, training, and evaluation is available at \url{https://github.com/abtinpo/implementing-transformers}.

%-------------------------------------------------------------------------
\section{Transformer Architecture and Model Design}
\label{sec:architecture}

This section describes the architecture of the Transformer model implemented in this work.

\subsection{Input Embeddings and Tokenization}
Tokenization transforms raw text into discrete tokens. We implement Byte-Pair Encoding (BPE)~\cite{sennrich2016neural}, which iteratively merges frequent character pairs. Our vocabulary is constructed from the German-English parallel corpus. Special tokens are added for padding (\texttt{[PAD]}), unknown words (\texttt{[UNK]}), sequence start (\texttt{[BOS]}), and sequence end (\texttt{[EOS]}). The final vocabulary size is 32,000.

\subsection{Attention Mechanism}
Self-attention enables the model to weigh the relevance of different positions. We implement Scaled Dot-Product Attention:
\begin{equation}
\text{Attention}(Q, K, V) = \text{softmax}\left(\frac{QK^T}{\sqrt{d_k}}\right)V
\end{equation}
Multi-Head Attention (MHA) projects queries, keys, and values into multiple subspaces ($h=8$). In the decoder, we utilize masking to prevent positions from attending to subsequent positions (autoregressive property) and cross-attention to attend to encoder outputs.

\subsection{Sinusoidal Positional Encoding}
To inject order information, we implement sinusoidal positional encoding:
\begin{equation}
PE_{(pos, 2i)} = \sin\left(\frac{pos}{10000^{2i/d_{model}}}\right)
\end{equation}
This allows the model to learn relative positions via linear transformations.

\subsection{Position-wise Feed-Forward Network}
Each layer contains a feed-forward network defined as:
\begin{equation}
\text{FFN}(x) = \max(0, xW_1 + b_1)W_2 + b_2
\end{equation}
We use an inner dimension of $d_{ff} = 2048$.

\subsection{Layer Normalization and Residuals}
To stabilize training and improve gradient flow, residual connections surround each sub-layer, followed by layer normalisation. Given an input vector $x$, layer normalization is defined as:
\begin{equation}
\text{LayerNorm}(x) = \frac{x - \mu}{\sqrt{\sigma^2 + \epsilon}} \cdot \gamma + \beta
\end{equation}
where $\mu$ and $\sigma^2$ denote the mean and variance computed across feature dimensions. The final output of a sub-layer is $\text{LayerNorm}(x + \text{Sublayer}(x))$.

\subsection{Concrete Model Configuration}
The model consists of $N=6$ encoder and decoder layers with $d_{model} = 512$.

\subsection{Parameter Sharing}
Following the original Transformer, we share parameters across three embedding matrices: the encoder input embedding, the decoder input embedding, and the pre-softmax linear transformation. This reduces the model size significantly---instead of three separate matrices of size $V \times d_{model}$, only one is needed. For $V=32000$ and $d_{model}=512$, this saves approximately 33M parameters. Beyond memory savings, parameter sharing enforces semantic consistency: words maintain the same representation whether they appear in the source or target sequence, improving generalization and providing stronger learning signals for rare words.

%-------------------------------------------------------------------------
\section{Training Setup and Optimization}
\label{sec:training}

\subsection{Dataset and Configuration}
We train on the WMT17 German-English dataset. The data is preprocessed with BPE tokenization. We use 5,906,184 training pairs. The model is trained for 5 epochs with a batch size of 64 on a single NVIDIA H100 GPU (96GB VRAM).

\subsection{Optimization}
We use the AdamW optimizer, which decouples weight decay from gradient updates. The optimizer computes biased first- and second-moment estimates:
\begin{equation}
m_t = \beta_1 m_{t-1} + (1 - \beta_1) g_t
\end{equation}
\begin{equation}
v_t = \beta_2 v_{t-1} + (1 - \beta_2) g_t^2
\end{equation}
Since $m_0 = v_0 = 0$, early estimates are biased toward zero. Bias-corrected estimates are computed as $\hat{m}_t = m_t / (1 - \beta_1^t)$ and $\hat{v}_t = v_t / (1 - \beta_2^t)$. AdamW applies weight decay directly to parameters rather than adding it to the gradient, ensuring consistent regularization regardless of gradient magnitude. We use $\beta_1 = 0.9$, $\beta_2 = 0.98$, $\epsilon = 10^{-9}$, and weight decay of $0.1$ (excluding bias and layer normalization parameters).

We apply a custom learning rate schedule with linear warmup ($warmup = 4000$) and inverse square root decay:
\begin{equation}
lr(t) = d_{model}^{-0.5} \cdot \min(t^{-0.5}, t \cdot warmup^{-1.5})
\end{equation}
This allows the optimizer to adapt to gradient scales early in training.

\subsection{Regularisation and Stability}
We apply dropout ($P_{drop} = 0.1$) to prevent overfitting. We employ mixed-precision training using BFloat16 to accelerate computation while maintaining numerical stability. Additionally, we apply gradient clipping with a maximum norm of 1.0 to prevent exploding gradients and ensure stable training dynamics.

%-------------------------------------------------------------------------
\section{Grouped Query Attention (Extension)}
\label{sec:gqa}

Standard MHA creates a memory bottleneck during inference due to the size of the KV cache. To address this, we extend our implementation with Grouped Query Attention (GQA) \cite{ainslie2023gqa}. GQA interpolates between MHA and Multi-Query Attention (MQA). Query heads are divided into $G$ groups, sharing a single key-value head per group.
\begin{itemize}
    \item \textbf{MHA:} $G=h$ (Highest quality, largest memory).
    \item \textbf{GQA:} $1 < G < h$ (Balanced).
    \item \textbf{MQA:} $G=1$ (Lowest memory, potential quality loss).
\end{itemize}
For our experiments, we focus on $G=2$, representing a 4$\times$ reduction in KV cache compared to the baseline ($h=8$).

%-------------------------------------------------------------------------
\section{Experiments and Results}
\label{sec:results}

\subsection{Experimental Setup}
All experiments are conducted on the WMT17 German-English dataset. Both the baseline and GQA models use the same architecture, optimizer, and learning rate schedule to ensure comparability.

\subsection{Baseline Results}
We evaluate translation quality using BLEU scores. The baseline MHA model achieves a score of 26.99. Figures~\ref{fig:train_loss} and~\ref{fig:val_loss} illustrate the training dynamics for both models.

\begin{figure*}[t]
\centering
\begin{subfigure}[b]{0.48\textwidth}
    \includegraphics[width=\linewidth]{plots/train_loss_MHA.png}
    \caption{MHA Training Loss}
\end{subfigure}
\hfill
\begin{subfigure}[b]{0.48\textwidth}
    \includegraphics[width=\linewidth]{plots/train_loss_GQA.png}
    \caption{GQA Training Loss}
\end{subfigure}
\caption{Training loss curves over approximately 460,000 steps. Both models show similar convergence patterns.}
\label{fig:train_loss}
\end{figure*}

\begin{figure*}[t]
\centering
\begin{subfigure}[b]{0.48\textwidth}
    \includegraphics[width=\linewidth]{plots/val_loss_MHA.png}
    \caption{MHA Validation Loss}
\end{subfigure}
\hfill
\begin{subfigure}[b]{0.48\textwidth}
    \includegraphics[width=\linewidth]{plots/val_loss_GQA.png}
    \caption{GQA Validation Loss}
\end{subfigure}
\caption{Validation loss curves showing similar convergence behavior for both attention mechanisms.}
\label{fig:val_loss}
\end{figure*}

\begin{figure*}[t]
\centering
\begin{subfigure}[b]{0.48\textwidth}
    \includegraphics[width=\linewidth]{plots/bleu_score_MHA.png}
    \caption{MHA BLEU Score}
\end{subfigure}
\hfill
\begin{subfigure}[b]{0.48\textwidth}
    \includegraphics[width=\linewidth]{plots/bleu_score_GQA.png}
    \caption{GQA BLEU Score}
\end{subfigure}
\caption{BLEU score progression on validation set during training. GQA achieves slightly higher final BLEU.}
\label{fig:bleu_curve}
\end{figure*}

\subsection{GQA Extension Results}
Table~\ref{tab:bleu} compares the baseline against the GQA variant. GQA ($G=2$) achieves a competitive BLEU score of 27.05, slightly outperforming the baseline while using only 25\% of the KV cache memory.

\begin{table}[h]
\centering
\begin{tabular}{lccc}
\toprule
\textbf{Model} & \textbf{BLEU} & \textbf{Cache Size} & \textbf{Parameters} \\
\midrule
MHA (Base) & 26.99 & 1.0$\times$ & 60.5M \\
GQA ($G=2$) & 27.05 & 0.25$\times$ & 53.4M \\
\bottomrule
\end{tabular}
\caption{BLEU scores, relative KV cache sizes, and model parameters.}
\label{tab:bleu}
\end{table}

\subsection{Convergence Analysis}
We observed that GQA models converge with similar dynamics to the baseline. The validation loss curves (Figure~\ref{fig:val_loss}) show that both models reach comparable final losses, indicating that the reduced KV head count does not significantly impact model capacity for this task.

%-------------------------------------------------------------------------
\section{Qualitative Analysis of Translations}
\label{sec:qualitative}

Table~\ref{tab:examples} presents representative translation examples. We analyze fluency, semantic adequacy, and recurring error patterns.

Interestingly, GQA ($G=2$) often produces more fluent and accurate translations than the MHA baseline, despite using fewer parameters and reduced KV cache.

\textbf{Semantic Accuracy.} Both models occasionally struggle with domain-specific terms. For example, the German word ``Leichnam'' (corpse/body) is sometimes left untranslated. However, GQA demonstrates better handling of compound nouns and professional terminology (e.g., correctly translating ``Koch'' as ``chef'' rather than ``cook'').

\textbf{Syntactic Fluency.} GQA shows improved fluency in relative clause constructions. In the complex sentence example, GQA correctly produces ``who recently moved'' while MHA generates the awkward ``recently drawn to.''

\textbf{Long vs. Short Sentences.} Both models perform excellently on short sentences, often achieving perfect translations. On longer sentences, we observe occasional repetition artifacts in failure cases (e.g., ``to be a city to be a city...''), which may indicate attention degradation on very long sequences.

\begin{table}[h]
\centering
\small
\begin{tabular}{p{2.2cm}p{2.4cm}p{2.4cm}}
\toprule
\textbf{Source (DE)} & \textbf{MHA Output} & \textbf{GQA Output} \\
\midrule
28-jähriger Koch in San Francisco Mall tot aufgefunden & 28 years of cook in San Francisco Mall dead & 28-year-old chef found dead in San Francisco Mall \\
\midrule
Ein 28-jähriger Koch, der vor kurzem nach San Francisco gezogen ist... & A 28 year-old kitchen recently drawn to San Francisco... & A 28-year-old chef, who recently moved to San Francisco... \\
\bottomrule
\end{tabular}
\caption{Qualitative comparison of translations. GQA produces more fluent output.}
\label{tab:examples}
\end{table}

%-------------------------------------------------------------------------
\section{Lessons Learned and Challenges}
\label{sec:lessons}

\textbf{Python and CUDA Compatibility.} A significant challenge arose from Python version incompatibility with the H100 GPU and PyTorch. Initially, using Python 3.9 caused the model predictions to collapse to a single repeated token during inference. Upgrading to Python 3.11 resolved this issue, highlighting the importance of ensuring compatibility between hardware, CUDA drivers, and software dependencies.

\textbf{Numerical Stability in Masked Attention.} When implementing masked self-attention, using \texttt{float('-inf')} for masked positions led to NaN values during the softmax computation, particularly when entire rows were masked. Replacing \texttt{-inf} with a large negative constant ($-10^9$) resolved this issue while maintaining the desired masking behavior.

\textbf{GQA Broadcasting.} Implementing GQA required careful handling of tensor broadcasting when expanding the reduced KV heads to match query heads. Ensuring correct dimension alignment between $Q \in \mathbb{R}^{B \times h \times L \times d_k}$ and $K, V \in \mathbb{R}^{B \times g \times L \times d_k}$ was essential for correct attention computation.

%-------------------------------------------------------------------------
\section{Discussion and Insights}
\label{sec:discussion}

The primary trade-off investigated in this work is between memory efficiency and translation quality.

\textbf{Memory Efficiency.} As shown in Table~\ref{tab:memory}, GQA significantly reduces the memory footprint.

\begin{table}[h]
\centering
\begin{tabular}{lcc}
\toprule
\textbf{Model} & \textbf{KV Heads} & \textbf{Relative Mem} \\
\midrule
MHA & 8 & 1.0$\times$ \\
GQA ($G=2$) & 2 & 0.25$\times$ \\
\bottomrule
\end{tabular}
\caption{Memory requirements.}
\label{tab:memory}
\end{table}

\textbf{Inference Speed.} We analyzed decoding latency across different batch sizes (Table~\ref{tab:latency}). Reduced memory bandwidth usage in GQA translates to faster token generation, particularly at higher batch sizes where memory bandwidth is the bottleneck.

\begin{table}[h]
\centering
\begin{tabular}{lccc}
\toprule
\multirow{2}{*}{\textbf{Model}} & \multicolumn{3}{c}{\textbf{Latency (ms)}} \\
& \textbf{Batch 1} & \textbf{Batch 32} & \textbf{Batch 128} \\
\midrule
MHA & 7.06 & 14.91 & 52.40 \\
GQA ($G=2$) & 7.73 & 13.61 & 47.57 \\
\bottomrule
\end{tabular}
\caption{Inference latency comparison at different batch sizes. GQA shows improved throughput at higher batch sizes due to reduced memory bandwidth requirements.}
\label{tab:latency}
\end{table}

%-------------------------------------------------------------------------
\section{Conclusion}
\label{sec:conclusion}

We successfully implemented a Transformer for German-to-English translation and investigated Grouped Query Attention. Our results show that GQA ($G=2$) not only matches but slightly exceeds baseline quality (27.05 vs 26.99 BLEU) while reducing KV cache memory by 75\% and model parameters by 12\%. This highlights GQA as an effective strategy for deploying Transformers in resource-constrained environments without sacrificing translation quality.

%%%%%%%%% REFERENCES
{
    \small
    \bibliographystyle{ieeenat_fullname}
    \bibliography{main}
}

\end{document}